% Options for packages loaded elsewhere
\PassOptionsToPackage{unicode}{hyperref}
\PassOptionsToPackage{hyphens}{url}
\PassOptionsToPackage{dvipsnames,svgnames,x11names}{xcolor}
%
\documentclass[
  letterpaper,
  DIV=11,
  numbers=noendperiod]{scrartcl}

\usepackage{amsmath,amssymb}
\usepackage{iftex}
\ifPDFTeX
  \usepackage[T1]{fontenc}
  \usepackage[utf8]{inputenc}
  \usepackage{textcomp} % provide euro and other symbols
\else % if luatex or xetex
  \usepackage{unicode-math}
  \defaultfontfeatures{Scale=MatchLowercase}
  \defaultfontfeatures[\rmfamily]{Ligatures=TeX,Scale=1}
\fi
\usepackage{lmodern}
\ifPDFTeX\else  
    % xetex/luatex font selection
\fi
% Use upquote if available, for straight quotes in verbatim environments
\IfFileExists{upquote.sty}{\usepackage{upquote}}{}
\IfFileExists{microtype.sty}{% use microtype if available
  \usepackage[]{microtype}
  \UseMicrotypeSet[protrusion]{basicmath} % disable protrusion for tt fonts
}{}
\makeatletter
\@ifundefined{KOMAClassName}{% if non-KOMA class
  \IfFileExists{parskip.sty}{%
    \usepackage{parskip}
  }{% else
    \setlength{\parindent}{0pt}
    \setlength{\parskip}{6pt plus 2pt minus 1pt}}
}{% if KOMA class
  \KOMAoptions{parskip=half}}
\makeatother
\usepackage{xcolor}
\setlength{\emergencystretch}{3em} % prevent overfull lines
\setcounter{secnumdepth}{5}
% Make \paragraph and \subparagraph free-standing
\makeatletter
\ifx\paragraph\undefined\else
  \let\oldparagraph\paragraph
  \renewcommand{\paragraph}{
    \@ifstar
      \xxxParagraphStar
      \xxxParagraphNoStar
  }
  \newcommand{\xxxParagraphStar}[1]{\oldparagraph*{#1}\mbox{}}
  \newcommand{\xxxParagraphNoStar}[1]{\oldparagraph{#1}\mbox{}}
\fi
\ifx\subparagraph\undefined\else
  \let\oldsubparagraph\subparagraph
  \renewcommand{\subparagraph}{
    \@ifstar
      \xxxSubParagraphStar
      \xxxSubParagraphNoStar
  }
  \newcommand{\xxxSubParagraphStar}[1]{\oldsubparagraph*{#1}\mbox{}}
  \newcommand{\xxxSubParagraphNoStar}[1]{\oldsubparagraph{#1}\mbox{}}
\fi
\makeatother


\providecommand{\tightlist}{%
  \setlength{\itemsep}{0pt}\setlength{\parskip}{0pt}}\usepackage{longtable,booktabs,array}
\usepackage{calc} % for calculating minipage widths
% Correct order of tables after \paragraph or \subparagraph
\usepackage{etoolbox}
\makeatletter
\patchcmd\longtable{\par}{\if@noskipsec\mbox{}\fi\par}{}{}
\makeatother
% Allow footnotes in longtable head/foot
\IfFileExists{footnotehyper.sty}{\usepackage{footnotehyper}}{\usepackage{footnote}}
\makesavenoteenv{longtable}
\usepackage{graphicx}
\makeatletter
\def\maxwidth{\ifdim\Gin@nat@width>\linewidth\linewidth\else\Gin@nat@width\fi}
\def\maxheight{\ifdim\Gin@nat@height>\textheight\textheight\else\Gin@nat@height\fi}
\makeatother
% Scale images if necessary, so that they will not overflow the page
% margins by default, and it is still possible to overwrite the defaults
% using explicit options in \includegraphics[width, height, ...]{}
\setkeys{Gin}{width=\maxwidth,height=\maxheight,keepaspectratio}
% Set default figure placement to htbp
\makeatletter
\def\fps@figure{htbp}
\makeatother
% definitions for citeproc citations
\NewDocumentCommand\citeproctext{}{}
\NewDocumentCommand\citeproc{mm}{%
  \begingroup\def\citeproctext{#2}\cite{#1}\endgroup}
\makeatletter
 % allow citations to break across lines
 \let\@cite@ofmt\@firstofone
 % avoid brackets around text for \cite:
 \def\@biblabel#1{}
 \def\@cite#1#2{{#1\if@tempswa , #2\fi}}
\makeatother
\newlength{\cslhangindent}
\setlength{\cslhangindent}{1.5em}
\newlength{\csllabelwidth}
\setlength{\csllabelwidth}{3em}
\newenvironment{CSLReferences}[2] % #1 hanging-indent, #2 entry-spacing
 {\begin{list}{}{%
  \setlength{\itemindent}{0pt}
  \setlength{\leftmargin}{0pt}
  \setlength{\parsep}{0pt}
  % turn on hanging indent if param 1 is 1
  \ifodd #1
   \setlength{\leftmargin}{\cslhangindent}
   \setlength{\itemindent}{-1\cslhangindent}
  \fi
  % set entry spacing
  \setlength{\itemsep}{#2\baselineskip}}}
 {\end{list}}
\usepackage{calc}
\newcommand{\CSLBlock}[1]{\hfill\break\parbox[t]{\linewidth}{\strut\ignorespaces#1\strut}}
\newcommand{\CSLLeftMargin}[1]{\parbox[t]{\csllabelwidth}{\strut#1\strut}}
\newcommand{\CSLRightInline}[1]{\parbox[t]{\linewidth - \csllabelwidth}{\strut#1\strut}}
\newcommand{\CSLIndent}[1]{\hspace{\cslhangindent}#1}

\KOMAoption{captions}{tableheading}
\makeatletter
\@ifpackageloaded{caption}{}{\usepackage{caption}}
\AtBeginDocument{%
\ifdefined\contentsname
  \renewcommand*\contentsname{Table of contents}
\else
  \newcommand\contentsname{Table of contents}
\fi
\ifdefined\listfigurename
  \renewcommand*\listfigurename{List of Figures}
\else
  \newcommand\listfigurename{List of Figures}
\fi
\ifdefined\listtablename
  \renewcommand*\listtablename{List of Tables}
\else
  \newcommand\listtablename{List of Tables}
\fi
\ifdefined\figurename
  \renewcommand*\figurename{Figure}
\else
  \newcommand\figurename{Figure}
\fi
\ifdefined\tablename
  \renewcommand*\tablename{Table}
\else
  \newcommand\tablename{Table}
\fi
}
\@ifpackageloaded{float}{}{\usepackage{float}}
\floatstyle{ruled}
\@ifundefined{c@chapter}{\newfloat{codelisting}{h}{lop}}{\newfloat{codelisting}{h}{lop}[chapter]}
\floatname{codelisting}{Listing}
\newcommand*\listoflistings{\listof{codelisting}{List of Listings}}
\makeatother
\makeatletter
\makeatother
\makeatletter
\@ifpackageloaded{caption}{}{\usepackage{caption}}
\@ifpackageloaded{subcaption}{}{\usepackage{subcaption}}
\makeatother

\ifLuaTeX
  \usepackage{selnolig}  % disable illegal ligatures
\fi
\usepackage{bookmark}

\IfFileExists{xurl.sty}{\usepackage{xurl}}{} % add URL line breaks if available
\urlstyle{same} % disable monospaced font for URLs
\hypersetup{
  pdftitle={Greatness},
  pdfauthor={First author; Another author},
  colorlinks=true,
  linkcolor={blue},
  filecolor={Maroon},
  citecolor={Blue},
  urlcolor={Blue},
  pdfcreator={LaTeX via pandoc}}


\title{Greatness\thanks{Code and data are available at:
\url{https://github.com/RohanAlexander/starter_folder}.}}
\usepackage{etoolbox}
\makeatletter
\providecommand{\subtitle}[1]{% add subtitle to \maketitle
  \apptocmd{\@title}{\par {\large #1 \par}}{}{}
}
\makeatother
\subtitle{Why Mahomes could be the GOAT}
\author{First author \and Another author}
\date{November 21, 2024}

\begin{document}
\maketitle
\begin{abstract}
First sentence. Second sentence. Third sentence. Fourth sentence.
\end{abstract}


It is widely accepted the Tom Brady is the greatest football player of
all time 7 superbowls etc In recent years, and anecdotes that I have
personally witnessed suggests that maybe there is an argument the
mahomes could become the goat.

\section{Introduction}\label{introduction}

Overview paragraph

Estimand paragraph

Results paragraph

Why it matters paragraph

Telegraphing paragraph: The remainder of this paper is structured as
follows. Section~\ref{sec-data}\ldots.

\section{Data}\label{sec-data}

\subsection{Overview}\label{overview}

We use the statistical programming language R (R Core Team 2023)\ldots.
Our data (Toronto Shelter \& Support Services 2024)\ldots. Following
Alexander (2023), we consider\ldots{}

Overview text

\subsection{Measurement}\label{measurement}

Some paragraphs about how we go from a phenomena in the world to an
entry in the dataset.

\subsection{Outcome variables}\label{outcome-variables}

Add graphs, tables and text. Use sub-sub-headings for each outcome
variable or update the subheading to be singular.

Some of our data is of penguins (\textbf{?@fig-bills}), from Horst,
Hill, and Gorman (2020).

Talk more about it.

And also planes (\textbf{?@fig-planes}). (You can change the height and
width, but don't worry about doing that until you have finished every
other aspect of the paper - Quarto will try to make it look nice and the
defaults usually work well once you have enough text.)

\subsection{Predictor variables}\label{predictor-variables}

Add graphs, tables and text.

Use sub-sub-headings for each outcome variable and feel free to combine
a few into one if they go together naturally.

\subsection{Model Overview}\label{model-overview}

We designed a predictive model to estimate Patrick Mahomes' cumulative
career statistics over an additional 200 games. The model leverages
historical game data from Tom Brady as a benchmark and Patrick Mahomes'
existing career data to generate predictions for key performance
metrics.

\subsubsection{Mathematical Notation}\label{mathematical-notation}

The model is a multivariate linear regression defined as:

\[
\hat{y}_i = \beta_0 + \sum_{j=1}^p \beta_j x_{ij} + \epsilon_i
\]

Where:

\begin{itemize}
\tightlist
\item
  \$ \hat{y}\_i \$: Predicted value of the response variable (e.g.,
  total passing yards).\\
\item
  \$ \beta\_0 \$: Intercept term.\\
\item
  \$ \beta\emph{j \$: Coefficients for predictors \$ x}\{ij\} \$, which
  represent the influence of the \$ j \$-th feature on the response
  variable.\\
\item
  \$ x\_\{ij\} \$: Observed value of the \$ j \$-th predictor for the \$
  i \$-th observation.\\
\item
  \$ \epsilon\_i \$: Residual error term, assumed to be normally
  distributed with mean 0.
\end{itemize}

We estimate \$ \beta\_0 \$ and \$ \beta\_j \$ using Ordinary Least
Squares (OLS) on Tom Brady's historical data, as his career provides a
well-documented trajectory for long-term performance.

\subsubsection{Variables and
Justification}\label{variables-and-justification}

The predictors (\$ x\_j \$) included in the model are chosen based on
their relevance to a quarterback's performance:

\begin{enumerate}
\def\labelenumi{\arabic{enumi}.}
\tightlist
\item
  \textbf{Passing Metrics}:

  \begin{itemize}
  \tightlist
  \item
    \texttt{completions}, \texttt{attempts}, \texttt{passing\_yards},
    \texttt{passing\_tds}, \texttt{interceptions},
    \texttt{passing\_air\_yards}, \texttt{passing\_first\_downs}.\\
  \item
    These metrics directly influence a quarterback's overall
    contribution to team success.
  \end{itemize}
\item
  \textbf{Rushing Metrics}:

  \begin{itemize}
  \tightlist
  \item
    \texttt{carries}, \texttt{rushing\_yards}, \texttt{rushing\_tds},
    \texttt{rushing\_first\_downs}.\\
  \item
    While not the primary focus, rushing performance is crucial for
    dual-threat quarterbacks like Mahomes.
  \end{itemize}
\item
  \textbf{Ball Security}:

  \begin{itemize}
  \tightlist
  \item
    \texttt{sack\_fumbles}, \texttt{rushing\_fumbles}.\\
  \item
    Turnovers are critical in evaluating overall impact.
  \end{itemize}
\end{enumerate}

These features ensure the model captures both primary and auxiliary
aspects of quarterback performance.

\subsubsection{Model Assumptions}\label{model-assumptions}

\begin{enumerate}
\def\labelenumi{\arabic{enumi}.}
\tightlist
\item
  \textbf{Linearity}: The relationship between predictors and response
  variables is linear.\\
\item
  \textbf{Independence}: Residuals are independent across
  observations.\\
\item
  \textbf{Homoscedasticity}: Residuals have constant variance.\\
\item
  \textbf{Normality}: Residuals are normally distributed.\\
\item
  \textbf{Stationarity}: Mahomes' performance trajectory will follow
  trends observed in Brady's career.
\end{enumerate}

\subsubsection{Software and
Implementation}\label{software-and-implementation}

The model was implemented using R, leveraging the \texttt{lm} function
from the \texttt{stats} package for regression. The dataset was
preprocessed using \texttt{tidyverse}, and predictions were calculated
for an additional 200 games using the coefficients estimated from
Brady's data.

\subsubsection{Validation and
Diagnostics}\label{validation-and-diagnostics}

\begin{enumerate}
\def\labelenumi{\arabic{enumi}.}
\tightlist
\item
  \textbf{Train-Test Split}: The Brady dataset was split into training
  (80\%) and testing (20\%) sets. The model was trained on the training
  set and validated on the testing set.\\
\item
  \textbf{Error Metrics}:

  \begin{itemize}
  \tightlist
  \item
    \textbf{Root Mean Squared Error (RMSE)}: Measures the model's
    predictive accuracy.\\
  \item
    \textbf{Mean Absolute Error (MAE)}: Evaluates average prediction
    error.\\
  \end{itemize}
\item
  \textbf{Residual Analysis}:

  \begin{itemize}
  \tightlist
  \item
    Residual plots confirmed the assumptions of homoscedasticity and
    normality.\\
  \end{itemize}
\item
  \textbf{Out-of-Sample Testing}: Predicted Mahomes' statistics on his
  observed data and compared with actual results to ensure alignment.
\end{enumerate}

\subsubsection{Alternative Models
Considered}\label{alternative-models-considered}

\begin{enumerate}
\def\labelenumi{\arabic{enumi}.}
\tightlist
\item
  \textbf{Decision Trees}:

  \begin{itemize}
  \tightlist
  \item
    Strengths: Captures non-linear relationships.\\
  \item
    Weaknesses: Tends to overfit without pruning; less interpretable
    than linear regression.
  \end{itemize}
\item
  \textbf{Bayesian Regression}:

  \begin{itemize}
  \tightlist
  \item
    Strengths: Allows incorporation of priors, producing probabilistic
    predictions.\\
  \item
    Weaknesses: Increased complexity and computational requirements.
  \end{itemize}
\item
  \textbf{Final Choice}:

  \begin{itemize}
  \tightlist
  \item
    Linear regression was chosen for its balance of simplicity,
    interpretability, and performance. It aligns with the assumption
    that a quarterback's performance trends over time can be captured
    linearly.
  \end{itemize}
\end{enumerate}

\subsubsection{Limitations}\label{limitations}

\begin{enumerate}
\def\labelenumi{\arabic{enumi}.}
\tightlist
\item
  \textbf{Career Longevity}:

  \begin{itemize}
  \tightlist
  \item
    The model assumes Mahomes will continue performing at a consistent
    level for 200 games. This may not account for potential injuries,
    performance decline, or external factors.\\
  \end{itemize}
\item
  \textbf{Sample Bias}:

  \begin{itemize}
  \tightlist
  \item
    Relying on Brady's career as a benchmark may introduce bias, as it
    assumes Mahomes will follow a similar trajectory.\\
  \end{itemize}
\item
  \textbf{Feature Engineering}:

  \begin{itemize}
  \tightlist
  \item
    Excluding contextual factors like team strength or play style may
    limit predictive accuracy.
  \end{itemize}
\end{enumerate}

\subsubsection{Conclusion}\label{conclusion}

The linear regression model provides a robust yet interpretable
framework for predicting Mahomes' lifetime stats. It is grounded in
historical data and validated through residual analysis and error
metrics. While limitations exist, the model effectively captures the
essence of Mahomes' expected performance trajectory based on Brady's
historical patterns.

\section{Results}\label{results}

\newpage

\section*{References}\label{references}
\addcontentsline{toc}{section}{References}

\phantomsection\label{refs}
\begin{CSLReferences}{1}{0}
\bibitem[\citeproctext]{ref-tellingstories}
Alexander, Rohan. 2023. \emph{Telling Stories with Data}. Chapman;
Hall/CRC. \url{https://tellingstorieswithdata.com/}.

\bibitem[\citeproctext]{ref-palmerpenguins}
Horst, Allison Marie, Alison Presmanes Hill, and Kristen B Gorman. 2020.
\emph{{palmerpenguins: Palmer Archipelago (Antarctica) penguin data}}.
\url{https://doi.org/10.5281/zenodo.3960218}.

\bibitem[\citeproctext]{ref-citeR}
R Core Team. 2023. \emph{{R: A Language and Environment for Statistical
Computing}}. Vienna, Austria: R Foundation for Statistical Computing.
\url{https://www.R-project.org/}.

\bibitem[\citeproctext]{ref-shelter}
Toronto Shelter \& Support Services. 2024. \emph{Deaths of Shelter
Residents}.
\url{https://open.toronto.ca/dataset/deaths-of-shelter-residents/}.

\end{CSLReferences}




\end{document}
